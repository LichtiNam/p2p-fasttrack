In diesem Abschnitt geht es darum auf die Probleme \ref{subsec:probT} von FastTrack einzugehen.
Außerdem wird im letzten Teil \ref{subsec:impl} angegeben, welche Implementierung es vom FastTrack Protokoll gibt und ein kurzer Überblick wie es damit weiter ging.


\subsection{Probleme von FastTrack}
\label{subsec:probT}

Eines der Probleme, welche aus dem Messungen von \cite{liang2006fasttrack}, wie in Abbildung \ref{fig:rtt} zu erkennen ist, dass die Round Trip Time von etwa 85 \% der SN zu SN Verbindungen bei maximal 100 ms beträgt.
Wie in der Abbildung \ref{fig:rttcaida} von Caida.org von 2001 \cite{caida} zu erkennen, beträgt die Round Trip Time von Amerika nach Europa und Asien etwa 200 ms.
Somit wird es schwierig das ganze Netzwerk zu durchsuchen, da nur wenige der SN über Verbindungen verfügen, welche weiter entfernt sind.

\begin{figure}
\includegraphics[scale=0.3]{gfx/dist_density_rie_20010513}
\caption{RTT Messung von Caida (Quelle \cite{caida})}
\label{fig:rttcaida}
\end{figure}


\subsubsection{UHASH}
\label{subsubsec:uhash}

Ein weiteres Problem ist der Content Hash Funktion UUHash \cite{uuHash} von FastTrack.
Der Hash kann schnell berechnet werden, da nur alle $2^n, n \in \mathbb{N}$ die ersten 300 KiB in die Berechnung einfließen.
Somit ist es möglich die Bereiche, welche nicht zur Hash Berechnung verwendet wurde mit Schadcode zu einzufügen.
Dies führte jedoch auch zu einem Problem, so das 2005 etwa 50\% aller Dateien infiziert waren. \cite{menneck2} 
Da viele MP3s über FastTrack getauscht werden und die Musikfirmen dies verhindern wollen, wurden viele MP3s kompromittiert um die Anwender zu schädigen. \cite{veitinger2002}


\subsection{Implementationen}
\label{subsec:impl}

Die offizielle Implementierung des FastTrack Netzwerkes ist Kazaa \cite{kazaa} von den Beiden FastTrack Entwicklern.

Eine verbreitete inoffizielle Implementierung ist die Kazaa-Lite \cite{kazaaLite} Version.
Diese verwendet bei der Suche nicht nur den Eltern SN sondern versucht an mehrere SNs seine Suchanfrage zu senden.
Dabei wird zuerst die Anfrage an den Eltern SN gesendet und auf die Antwort gewartet.
Wenn der Knoten die Antwort erhält schließt dieser die Verbindung und sendet die Anfrage an einen neuen SN.\cite{liang2006fasttrack}

Zwei andere Implementierungen, welche das FastTrack Netwerk benutzten sind Grokster \cite{grokster} und Morpheus \cite{morpheus}, wobei Morpheus 2002 auf das Gnutella \cite{gnutella} Netzwerk umgestiegen ist. \cite{morphvsKazaa}

Außerdem gibt es eine Implementierung für giFT \cite{gift}, dies ist eine Sammlung für Peer-To-Peer Protokolle und FastTrack wurde per reverse Engineering als Plugin hinzugefügt \cite{liang2006fasttrack}

Eine der bekanntesten Weiterentwicklung die auf FastTrack basiert und aufbaut ist Skype \cite{skypeAna}, welches von den selben Entwicklern entwickelt wurde.