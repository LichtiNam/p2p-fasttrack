\documentclass[ucs,9pt]{beamer}

% Copyright 2004 by Till Tantau <tantau@users.sourceforge.net>.
%
% In principle, this file can be redistributed and/or modified under
% the terms of the GNU Public License, version 2.
%
% However, this file is supposed to be a template to be modified
% for your own needs. For this reason, if you use this file as a
% template and not specifically distribute it as part of a another
% package/program, I grant the extra permission to freely copy and
% modify this file as you see fit and even to delete this copyright
% notice.
%
% Modified by Tobias G. Pfeiffer <tobias.pfeiffer@math.fu-berlin.de>
% to show usage of some features specific to the FU Berlin template.

% remove this line and the "ucs" option to the documentclass when your editor is not utf8-capable
\usepackage[utf8x]{inputenc}    % to make utf-8 input possible
\usepackage[german]{babel}     % hyphenation etc., alternatively use 'german' as parameter
\usepackage[inkscape={/usr/bin/inkscap‌​e -z -C}]{svg}
\usepackage{amsmath}
\include{fu-beamer-template}  % THIS is the line that includes the FU template!
\setbeamertemplate{navigation symbols}{} 
\usepackage{arev,t1enc} % looks nicer than the standard sans-serif font
% if you experience problems, comment out the line above and change
% the documentclass option "9pt" to "10pt"

% image to be shown on the title page (without file extension, should be pdf or png)
\titleimage{images/aufbau}

\title[FastTrack] % (optional, use only with long paper titles)
{FastTrack}

\subtitle
{}

\author[Author, Another] % (optional, use only with lots of authors)
{Martin Görick}
% - Give the names in the same order as the appear in the paper.

\institute[FU Berlin] % (optional, but mostly needed)
{Freie Universität Berlin}
% - Keep it simple, no one is interested in your street address.

\date[] % (optional, should be abbreviation of conference name)
{Seminar: Peer-To-Peer Overlay Network Systems}
% - Either use conference name or its abbreviation.
% - Not really informative to the audience, more for people (including
%   yourself) who are reading the slides online

\subject{}
% This is only inserted into the PDF information catalog. Can be left
% out.

% you can redefine the text shown in the footline. use a combination of
% \insertshortauthor, \insertshortinstitute, \insertshorttitle, \insertshortdate, ...
\renewcommand{\footlinetext}{\insertshortinstitute, \insertshorttitle, \insertshortdate}

% Delete this, if you do not want the table of contents to pop up at
% the beginning of each subsection:
%\AtBeginSection[]
%{
%  \begin{frame}<beamer>{Outline}
%    \tableofcontents[currentsection,hideothersubsections]
%  \end{frame}
%}

\begin{document}

\begin{frame}[plain]
  \titlepage
\end{frame}

\begin{frame}{Übersicht}
  \tableofcontents[]%hideothersubsections]
  % You might wish to add the option [pausesections]
\end{frame}

%\section{Introduction}

%\begin{frame}{title of frame} %title in brace
%\begin{itemize}
%\item test item
%\end{itemize}
%\end{frame}

\section{Allgemeine Information}

\begin{frame}{Allgemeine Informationen}
\begin{itemize}
	\item entwickelt von Niklas Zennstrom und Janus Friis
	\item proprietäres Protokoll
	\item normale Knoten (ON)
	\item super Knoten (SN)
	\item Windows Client (Kazaa)
	\item Ziel: Dateiaustausch
\end{itemize}
\end{frame}

\begin{frame}{Aufbau}
\begin{figure}
\includegraphics[scale=0.3]{images/aufbau}
\caption{Aufbau des FastTrack Netzwerkes}
\end{figure}
\end{frame}

\begin{frame}{Komponenten}
\begin{itemize}
	\item FastTrack Media Desktop
	\item Informationen in der Windows Registry
		\begin{itemize}
		\item SNs Cache
		\end{itemize}
	\item DDB Dateien
		\begin{itemize}
		\item Dateiname
		\item Dateigröße
		\item ContentHash
		\item Datei Beschreibungen
		\end{itemize}
	\item Dat Dateien
\end{itemize}
\end{frame}

\section{Funktionsweise}

\subsection{Nachrichten}

\begin{frame}{Nachrichten}
\begin{itemize}
\item Signalnachrichten, verschlüsselt
	\begin{itemize}
	\item Handshakes
	\item Meta Daten
	\item SN Liste
	\end{itemize}
\item Datei Transport
	\begin{itemize}
	\item direkte Verbindungen
	\item HTTP Nachrichten
	\end{itemize}
\item Gewerbliche Werbung
	\begin{itemize}
	\item HTTP Nachrichten
	\end{itemize}
\item Sofortnachrichten, verschlüsselt
\end{itemize}
\end{frame}

\begin{frame}{Format einer Signal Nachricht}
\begin{figure}
\includegraphics[scale=0.3]{images/signal_message}
\caption{Signal Nachricht Format}
\end{figure}
\end{frame}

\subsection{Join}

\begin{frame}{Join}
\begin{itemize}
\item UDP Pakete zu ausgewählte SNs
\item zu allen UDP Response eine TCP connection
	\begin{itemize}
	\item Schlüssel Austausch
	\item On sendet Peer information
		\begin{itemize}
		\item locale IP Adresse
		\item Port
		\item Username
		\end{itemize}
	\item SN sendet aktuelle Liste von SNs
	\item wählen eines SN
	\item kann Suchnachrichten senden
	\end{itemize}
\end{itemize}
\end{frame}

\begin{frame}{Join - Ablauf}
\begin{figure}
\includegraphics[scale=0.25]{images/join}
\caption{Ablauf eines Joins}
\end{figure}
\end{frame}

\begin{frame}{SN wählen}
\begin{itemize}
\item SN Cache
\begin{itemize}
\item IP Adresse
\item Port
\item Auslastung
\item Zeitstempel
\end{itemize}
\end{itemize}
\begin{itemize}
\item Auslastung
\item Lokalität
\end{itemize}
\end{frame}

\begin{frame}{RTT - Lokalitäten}
\begin{figure}
\includegraphics[scale=0.4]{images/rttSns}
\caption{Die RTT zwischen ON-SN und SN-SN}
\end{figure}
\end{frame}

\subsection{Search}

\begin{frame}{Suche}
\begin{itemize}
\item Suchanfrage an Eltern SN
\item Eltern SN durchsucht seine Metadaten
\item Eltern SN schickt gefundene Informationen zurück
\item Eltern SN schickt Suchanfrage an andere SNs
\end{itemize}
\end{frame}

\begin{frame}{Download über NAT}
\begin{itemize}
\item Peer A sendet Request zu SN von B
\item SN von B sendet Nachricht zu B 
\item B verbindet sich zu A
\end{itemize}
\end{frame}

\begin{frame}{Informationsaustausch SNs}
\begin{figure}
\includegraphics[scale=0.25]{images/SnToSn}
\caption{Austausch der Information zwischen SNs}
\end{figure}
\end{frame}

\section{Weiterführende Informationen}

\begin{frame}{Probleme}
\begin{itemize}
\item finden von Dateien
\begin{itemize}
\item RTT von SNs $< 100$
\end{itemize}
\item Dateihash schwach 
	\begin{itemize}
		\item UUHash
		\item 2005 etwa 50\% infiziert \cite{uuHash}
	\end{itemize}
\end{itemize}
\end{frame}

\begin{frame}{Implementierungen}
\begin{itemize}
\item Kazaa
\item Grokster
\item Morpheus (ab 2002 gnutella)
\item giFT (Open Source)
\end{itemize}
\begin{itemize}
\item Skype
\item Joost
\end{itemize}
\end{frame}

%\begin{frame}[fragile]
%  \frametitle{Test code frame}
%\begin{semiverbatim}
%Testcode
%\end{semiverbatim}
%\end{frame}

%\section{Test Section 2}

%\begin{frame}{}
%\begin{figure}
%\includegraphics[scale=0.5]{linux-logo}
%\caption{linux-logo}
%\end{figure}
%\end{frame}



% All of the following is optional and typically not needed. 
\appendix
\section<presentation>*{\appendixname}
\subsection<presentation>*{For Further Reading}

\begin{frame}[allowframebreaks]
  \frametitle<presentation>{For Further Reading}
    
  \begin{thebibliography}{10}

  \beamertemplatearticlebibitems
  % Followed by interesting articles. Keep the list short. 

  \bibitem{uuHash}
    Thomas Mennecke, Dec 2005, End of the Road for Overpeer 
    \newblock \url{http://www.slyck.com/story1019.html}
  \bibitem{fastTrackStudi}
    Jian Liang, Rakesh Kumar, Keith W. Ross, August 2005
    \newblock The FastTrack overlay: A measurement study    
  \end{thebibliography}
\end{frame}


\end{document}
